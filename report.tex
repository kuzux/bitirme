\documentclass[11pt]{article}
\usepackage[utf8]{inputenc}
\usepackage{times}
\usepackage{hyperref}
\usepackage{amsmath}
\usepackage{latexsym}
\usepackage{graphicx}
\usepackage{setspace}
\usepackage{multirow}

\onehalfspacing

\title{Visualization of Multivariate Time-Series Data \\ Midterm Report}
\author{Arda Çınar \\ Evin Pınar Örnek}
\date{March 2017}

\begin{document}

\maketitle

\section{Abstract} 

%% 

\section{Introduction}

\subsection{Motivation}

In the real world, there is a lot of data that needs to viewed by humans but simply is too complicated for humans to look at in an unprocessed form. Some form of visualization and data summarization can help immensely for viewing data that is too complicated to view. A fairly common form of this data is a set of data that is dependent on multiple variables one of which is time. There can be some interesting visualization methods applied under these extra conditions but there doesn't seem to be any matching these criteria.

\subsection{Example Data}
%% - Motivation, why are we doing this
%% - What type of data we have, what do we want to visualize

\section{Background}

%% - What is a tensor
%% - Our data and tensor relationship (tensor of rank 2 in R^3) 
%% - How are the tensors represented? -> bipartite graph(factor graphs)
%% - Data structures for tensors? 
%% - Visualization of 

\section{Methods}

%% Our method to represent tensor
%% Folding tensors, the operations 
%%  - Summation over an index (collapsing a tensor ;d)
%%  - Multiplication 
%% (Implementation)
%%  - How to upload data 
%%  - How did we extract the time series data
%%  - Libraries we use ... 

%%%%%% \section{Analysis}

\subsection {Prototype Implementation}

%% Graphs, outputs (haftaya haftasonu yapilabilir)

So, after all, to successfully visualize the data we have and make somewhat usable project out of it, we need to implement what we learned into a usable program. In the current day and age, the most widely-usable platform where our implementation can be used as easily as possible, plus considering that mobile users aren't our target, the web is the obvious option as the platform of choice. Not to mention that we are both familiar with the development environment from our previous experiences both within and outside of school.

\section {Conclusion} 

%% Simdiye kadar bunlari yaptik, 
%% Future work 

\end{document}
